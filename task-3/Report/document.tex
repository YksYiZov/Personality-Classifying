\documentclass{article}
\usepackage {xeCJK}
\usepackage{color}
\usepackage{ctex}
\usepackage{graphicx} 
\usepackage{float} 
\usepackage{enumitem}
\usepackage[hidelinks, colorlinks=True, linkcolor=black]{hyperref}
\usepackage{nicematrix}
\usepackage[linesnumbered,ruled,vlined]{algorithm2e}
\usepackage{tikz}
\usepackage{wrapfig}
\usepackage{makecell}
\usepackage{subfigure}
\usepackage{booktabs}
\usepackage{background}
\usepackage{amssymb}
\usepackage{booktabs}
\usepackage{listings}

\definecolor{section}{RGB}{102,205,170}

\title{\huge  自然语言处理\\ \large Project 1 Task 3}
\author{院系:人工智能学院\\姓名:王蔚昕\\学号:211300042}
\date{\today}
\graphicspath{{figures/}}
\setcounter{tocdepth}{2}
\setcounter{secnumdepth}{3}

\backgroundsetup{scale=1, angle=0, opacity=0.5, pages=some,contents={\includegraphics[height=\paperheight,width=\paperwidth]{D:/background.png}}}
\begin{document}
	\maketitle
	\newpage
	\tableofcontents
	\newpage
	\section{问题一}
	\subsection{实验方法}
	在这一环节,我使用了Kmeans进行新的划分,同时为了能与MBTI进行的性格分类进行划分,所以我依然把它分成了16种性格,以此来检验不同分类方法对于这一分类任务的影响。
	
	具体的来讲,我先用Bert-tiny得到原始数据的特征,随后使用输出的pooler\_output作为Kmeans的分类标准,将样本进行十六分类,得到新的标签。
	\subsection{实验结果}
	使用Kmeans成功完成了一个十六分类,同时根据这个十六分类的Kmeans对验证集也进行标签的更新。
	\section{问题二}
	\subsection{实验方法}
	结合上一问题中得到的新标签,和提取后的特征,我设计了一个线性分类网络,由四个线性层,一个Softmax层组成,用作分类器。
	\subsection{实验结果}
	\begin{table}[H]
		\begin{tabular}{ccc}
			\toprule
			使用模型 & 使用分类方法 & 准确率 \\
			\hline
			BertBase & MBTI & 40\% \\
			BertTiny & Kmeans & 70\%\\
			\bottomrule
		\end{tabular}
		\centering
	\end{table}
	\subsection{结果分析}
	显而易见的是,更复杂,能力更强的模型在MBTI上取得了更低的准确率,这显然是因为使用的分类方法的区别。
	\subsection{一些思考}
	\subsubsection{Kmeans得到的分类是否有意义}
	Kmeans一定可以对任何样本集完成一种分类,但是这种分类是否具有现实意义呢?我们知道Kmeans仅仅是对样本特征进行划分,找到一些“深层”联系,但是这些联系可能在实际生活中没有体现到,或者说,不能体现我们想要的信息。就本问题而言,被Kmeans分为同一类的很可能不是同一种性格,而是在语义上大意相同的语句,在我进行一些语句展示时证明了这一点,亦或者我本人不具备区分性格的能力。所以Kmeans得到的只是一个统计学上的分类,而并非是我们想要的性格分类。
	\subsubsection{为什么Kmeans得到的分类效果会好很多}
	Kmeans的分类标准是按照样本特征的距离来分类的,在这种情况下,同一类的样本的特征往往很相似,所以对于分类器而言,他只需要找到这些特征的分界线,也就是说Kmeans分类得到的结果几乎都是可分的。然而根据task-1中KNN的糟糕效果不难看出,对于大部分样本来讲,他在Bert编码后几乎是杂糅在一起的,只有只针对某一些类而言可能更容易完成分割。所以这可能降低了线性分类器分类的难度,导致Kmeans得到的分类效果更好,但是还是回到第一个问题,Kmeans的分类或许对应的是语义而非性格。
	\subsubsection{文本提取器可能的改进}
	有研究表明,LLM输出的特征信息越深层可能越接近语言的深层含义,表层特征更容易做错别字检测与语法检测,但是深层特征做这些效果不好,结合此前提出的猜测,语义往往和话题有关,而非性格,所以用来编码的特征应该更浅层,或者浅层和深层的结合,但是绝对不应该是最深层。为此,可能要用一些能够综合考量句子特点的模型来进行特征提取,而不是能够深入语义的模型。
	\section{对整个实验的回顾}
	\subsection{不足}
	在早期实验中,因为某些代码的错误,导致实验结果存在偏差,包括小数点位数的错误。
	
	同时,早期选择了错误的模型,导致训练难以在较大数据集上展开。
	
	由于一些技术设备问题,前期实验进行并不充分,数据及利用有限,导致某些结论来源于推测而非直接的实验证明。
	
	以上问题都会在我的github账户中进行更新完善。
	\subsection{收获}
	体验学习了面对nlp问题的处理方法,包括数据处理,模型训练等过程,对开源库transforms有了更多的了解和认识。
	
	通过调查文献与实验,对大语言模型的训练过程,推理过程有了更深层次的认识,同时借由此次文本分类任务,对分类器这一日常使用的简单部件有了深层的理解。
	
	这次项目锻炼了我进行问题探究的能力,受益匪浅。
\end{document}	
