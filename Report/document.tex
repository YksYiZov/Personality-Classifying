\documentclass{article}
\usepackage {xeCJK}
\usepackage{color}
\usepackage{ctex}
\usepackage{graphicx} 
\usepackage{float} 
\usepackage{enumitem}
\usepackage[hidelinks, colorlinks=True, linkcolor=black]{hyperref}
\usepackage{nicematrix}
\usepackage[linesnumbered,ruled,vlined]{algorithm2e}
\usepackage{tikz}
\usepackage{wrapfig}
\usepackage{makecell}
\usepackage{subfigure}
\usepackage{booktabs}
\usepackage{background}
\usepackage{amssymb}
\usepackage{listings}

\definecolor{section}{RGB}{102,205,170}

\title{\huge  自然语言处理\\ \large Project 1 Task 1}
\author{院系:人工智能学院\\姓名:王蔚昕\\学号:211300042}
\date{\today}
\graphicspath{{figures/}}
\setcounter{tocdepth}{2}
\setcounter{secnumdepth}{3}

\backgroundsetup{scale=1, angle=0, opacity=0.5, pages=some,contents={\includegraphics[height=\paperheight,width=\paperwidth]{D:/background.png}}}
\begin{document}
	\maketitle
	\newpage
	\tableofcontents
	\newpage
	\section{子问题1}
	\subsection{难点}
	\subsubsection{言语与性格的相关度}
	首先必须承认的是,言语一定与性格相关,但是在这里要说的事,言语与性格能有多相关?因为我认为语言对于性格来讲并非一个强有力的证明因素,他可能只是一个更容易收集到的因素,而且与性格有关联。所以我提出的猜想是,仅仅凭借言语可能不足以完全确定一个人的性格。一个人格可能有代表性行为,代表性处理事情的偏向,但是语言收到影响的东西太多了,同样的性格,在不同的环境中成长表现出来的言语可能是不同的,甚至截然相反。因为人们对于语言的学习大多数来自于身边的人,说话习惯也可能和身边的某个人或某些人有相似之处。所以我决定验证不同性格人之间言语的相似度以及同样性格人之间言语的相似度。
	\subsubsection{}
	
\end{document}	
